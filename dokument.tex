%%This is a very basic article template.
%%There is just one section and two subsections.
\documentclass[12pt, a4paper]{article}

\usepackage[utf8]{inputenc} % Kodowanie UTF8
\usepackage{polski} % Wsparcie dla PL
\usepackage{listings}
\usepackage{color}
\usepackage[usenames,dvipsnames,svgnames,table]{xcolor}
\usepackage{hyperref} % dla linkowania do stron www

\begin{document}
\title{Wprowadzenie do Windows Phone 7}
\author{Marek Lewandowski}
\date{\today}



\maketitle

\abstract{Celem artykułu jest zapoznanie czytelnika z niezbędnymi
zagadnieniami platformy WP7 oraz oszczędzenie mu potencjalnych
nieprzyjemności.}

\section{Disclaimer, czyli autor się kaja}
 Nie jestem specjalistą od Windows Phone'a, ani nawet nie znam C\# w stopniu
 większym niż podstawowy. Nie przeszkodziło mi to jednak, aby w dość krótkim czasie stworzyć 5 aplikacji na
konkurs Microsoftu i upublikować je na marketplace'sie. Swoją drogą konkurs
okazał się być niewypałem, ale to już temat na inny artykuł.

Chcę tutaj przedstawić niezbędne wiadomości, tak aby móc stworzyć aplikację,
która nie obleje certyfikacji co wiąże się ze sporą
stratą czasu, bo co najmniej 1 tygodnia, co czasem może być krytyczne.


\lstset{language=SQL, basicstyle=\footnotesize, numbers=left, numberstyle=\footnotesize, stepnumber=1, numbersep=10pt, breaklines=true, caption = opis kodu, frame=shadowbox, rulesepcolor=\color{SkyBlue}}
\begin{lstlisting}
kod
kod
kod
\end{lstlisting} 

\section{Słowo końcowe}
Koniec
\\\\\\
\emph{Źródła}:\\
\href{http://www.google.pl}{google}\\


\end{document}
