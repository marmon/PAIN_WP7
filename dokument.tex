%%This is a very basic article template.
%%There is just one section and two subsections.
\documentclass[12pt, a4paper]{article}

\usepackage[utf8]{inputenc} % Kodowanie UTF8
\usepackage{polski} % Wsparcie dla PL
\usepackage{listings}
\usepackage{color}
\usepackage[usenames,dvipsnames,svgnames,table]{xcolor}
\usepackage{hyperref} % dla linkowania do stron www

\begin{document}
\title{Wprowadzenie do Windows Phone 7}
\author{Marek Lewandowski}
\date{\today}



\maketitle

\abstract{Celem artykułu jest zapoznanie czytelnika z niezbędnymi
zagadnieniami platformy WP7 oraz oszczędzenie mu potencjalnych
nieprzyjemności.}

\section{Disclaimer, czyli autor się kaja}
 Nie jestem specjalistą od Windows Phone'a, ani nawet nie znam C\# w stopniu
 większym niż podstawowy. Nie przeszkodziło mi to jednak, aby w dość krótkim czasie stworzyć 5 aplikacji na
konkurs Microsoftu i upublikować je na marketplace'sie. Swoją drogą konkurs
okazał się być niewypałem, ale to już temat na inny artykuł.

Chcę tutaj przedstawić niezbędne wiadomości, tak aby móc stworzyć aplikację,
która nie obleje certyfikacji co wiąże się ze sporą
stratą czasu, bo co najmniej 1 tygodnia, co czasem może być krytyczne.

\section{Parę szybkich}
Czyli parę odpowiedzi na pytania, które możesz sobie teraz zadawać.
\subsection{Czy to coś kosztuje?}
Nie, SDK jest darmowe.
\subsection{Ile zapłacę za developerskie konto na marketplace?}
Nic, o ile jesteś studentem i skorzystasz z rejestracji przez program Microsoftu
\href{https://www.dreamspark.com/}{Dreamspark}. Od razu zaznaczę, że legitymacja
studencka nie wiele tutaj daje. Algorytm potwierdzenia statusu studenta wygląda
tak:
\begin{enumerate}
  \item Jak masz kartę ISIC to koniec, jak nie to 2.
  \item Znajdź kogoś z koła/grupy .NET żeby udostępnił Ci kod rejestracyjny, jak
  nie to 3.
  \item idź do 1.
\end{enumerate}
\subsection{Czy WP7 jest fajne?}
Tak, moim zdaniem jest całkiem niezłe, a na pewno bardzo proste do tworzenia nań
aplikacji. W miarę niezły emulator telefonu znacznie ułatwia pisanie aplikacji.
Nie jest to taki emulator jak ten do Androida, który uruchamia się minutę. Ten
jest całkiem szybki. Poza tym debugger Visuala jest chyba najlepszy, więc
pracuje się naprawdę przyjemnie. Poza tym mamy do dyspozycji takie coś jak
Expression Blender gdzie można dość łatwo składać UI i robić animację.
\subsection{\ldots serio?}
Tak, może tym razem przekona Cie to, że każdy telefon, na którym chodzi WP7 musi
zaspokajać
\href{http://msdn.microsoft.com/en-us/library/ff637514%28v=vs.92%29.aspx}{standard
 Microsoftu co do sprzętu}, więc nie trzeba pisać N wersji tej samej aplikacji,
 żeby ikonki się dobrze wyświetlały. Poza tym zakładając, że
zainteresowanie WP7 będzie rosnąć, a obecnie market jest raczej pusty to można
go opanować swoimi aplikacjami. Poza tym jest
\href{http://msdn.microsoft.com/en-us/library/ff402535%28v=vs.92%29}{MSDN} i
 można spokojnie brać przykłady metodą Copy'ego Paste'a i będą działać.


\lstset{language=SQL, basicstyle=\footnotesize, numbers=left, numberstyle=\footnotesize, stepnumber=1, numbersep=10pt, breaklines=true, caption = opis kodu, frame=shadowbox, rulesepcolor=\color{SkyBlue}}
\begin{lstlisting}
kod
kod
kod
\end{lstlisting} 

\section{Słowo końcowe}
Koniec
\\\\\\
\emph{Źródła}:\\
\href{http://www.google.pl}{google}\\


\end{document}
